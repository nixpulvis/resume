\documentclass[10pt,letterpaper]{article}
\usepackage[letterpaper,margin=3em]{geometry}
\usepackage{mdwlist}
\usepackage{color}
\usepackage{hyperref}

% Configuration.
\pagestyle{empty}
\setlength{\tabcolsep}{0em}

% indentsection style, used for sections that aren't already in lists
% that need indentation to the level of all text in the document
\newenvironment{indentsection}[1]
{\begin{list}{}
  {\setlength{\leftmargin}{#1}} \item[]
}
{\end{list}}

% opposite of above; bump a section back toward the left margin
\newenvironment{unindentsection}[1]
{\begin{list}{}
  {\setlength{\leftmargin}{-0.5#1}} \item[]
}
{\end{list}}
% format two pieces of text, one left aligned and one right aligned
\newcommand{\headerrow}[2]
{\begin{tabular*}{\linewidth}{l@{\extracolsep{\fill}}r}
  #1 &
  #2 \\
\end{tabular*}}

%%%%%%%%%%%%%%%%%%%%%%%%%%%%%%%%%%%%%%%%%%
% Here lies my resume in all it's glory. %
%%%%%%%%%%%%%%%%%%%%%%%%%%%%%%%%%%%%%%%%%%

\begin{document}

% Resume header.
\begin{center}
  \huge \textbf{Nathan Lilienthal} \\
  \large
  nathan@nixpulvis.com
  $\circ$
  202-701-4368
  $\circ$
  {\bf Available Now}
  \\
  github.com/nixpulvis
  $\circ$
  nixpulvis.github.io/nixpulvis/projects
  \vspace{-0.2em}
\end{center}


% Skills section.
\hrule
\begin{indentsection}{\parindent}
\begin{description*}
  \item[Languages:] Ruby, Rust, C/C++, LUA, and Racket. ECMAScript (JS),
    Python, Java, and Shell and many more...
  \item[Systems:] UNIX, Linux, Git, GitHub, Rails, Postgres, AVR/ARM, WoW
    (ask me about it).
\end{description*}
\end{indentsection}


% Experience section
\hrule
\vspace{-0.4em}
\subsection*{Professional Experience}
\begin{itemize}
  \parskip=0.1em

  \item
  \headerrow
    {\textbf{Northeastern University}}
    {\textbf{Boston, MA}}
  \headerrow
    {\emph{Research Programmer for IARPA's HECTOR Program}}
    {\emph{Aug. 2019 -- Aug. 2020}}
  \begin{itemize*}
    \item Collaberativly developed a hybrid-mode secure programming language
    design for MPC.
    \item Attended both remote and in-person technical exchange meetings with
    other teams of researchers.
    \item Built a prototype (currently insecure) implementation of our language
    forked from Rust.
    \item Began a formalism for our language(s), which will include sound
    typing rules, and reductions.
  \end{itemize*}
  \headerrow
    {\emph{Teaching Assistant for Fundamentals of Computer Science 1}}
    {\emph{Fall of 2012, 2013, 2014, and 2015}}
  \begin{itemize*}
    \item
    Conducted mini lectures, monitored class progress and answered students'
    questions.
    \item
    Discovered new ways to present concepts that facilitated student
    understanding.
    \item
    Held office hours to further help students with the course.
  \end{itemize*}

  \item
  \headerrow
    {\textbf{Forward Financing Inc.}}
    {\textbf{Boston, MA}}
  \headerrow
    {\emph{Sr. Software Engineer}}
    {\emph{May 2018 -- Aug. 2019}}
  \begin{itemize*}
    \item
    Developed a client wrapper for our Algolia search implementation.
    \item
    Performed many various application performance improvements, often caused
    by unacceptable response times, quickly.
    \item
    Planned and lead architecture refactoring and developer tool efforts,
    including object model improvements, an orchestration CLI for managing a
    complex Heroku + Salesforce microservice system, and more.
    \item
    Mentored our co-ops by providing deep code reviews, and spending time
    pairing on problems (both mine and theirs).
  \end{itemize*}

  \item
  \headerrow
    {\textbf{HOMER Energy}}
    {\textbf{Boulder, CO}}
  \headerrow
    {\emph{Software Developer, Summer Intern}}
    {\emph{Jul. 2017 - Nov. 2017, Summer 2012}}
  \begin{itemize*}
    \item
    Built API integrations for the HOMER C\# application, including REST
    and CSV file APIs. As a part of this process I refactored a lot of the
    code which imports data, and added tests.
    \item
    Developed an internal tool to view the Google Protocol Buffer used to pass
    values between all parts of the application. This allowed developers to
    quickly see what the values of program inputs and outputs were.
    \item
    Created a web based front-end for HOMER in Rails. This was the starting
    point for another version of HOMER, and provided a proof of concept for how
    to integrate the HOMER API with a webserver.
  \end{itemize*}

  \item
  \headerrow
    {\textbf{Apple Inc.}}
    {\textbf{Cupertino, CA}}
  \headerrow
    {\emph{Software Engineer}}
    {\emph{Jan. 2015 -- Aug. 2015, Jul. 2016 -- Jul. 2017}}
  \begin{itemize*}
    \item
    Developed a Ruby library (\texttt{radic}) and CLI (\texttt{radish}) for
    interacting with Apple's bug managment system (aka Radar).
    \item
    Worked on an internal tool for managing hardware validation. Somewhat
    inspired by Travis CI.
    \item
    Contributed to an internal tool for analysing pre-production device test
    data.
  \end{itemize*}

  \item
  \headerrow
    {\textbf{Americas Test Kitchen}}
    {\textbf{Boston, MA}}
  \headerrow
    {\emph{Web Developer}}
    {\emph{Jan. 2014 -- June 2014}}
  \begin{itemize*}
    \item
    Pushed code to the front-end and back-end for all four of Americas Test
    Kitchen's websites including bug fixes and technical infrastructure
    upgrades.
    \item
    Built modularized components to abstract functionality found common
    throughout the companies codebase.
  \end{itemize*}

  \item
  \headerrow
    {\textbf{Bluesocket - Adtran}}
    {\textbf{Burlington, MA}}
  \headerrow
    {\emph{Software Developer}}
    {\emph{Jan. 2013 -- June 2013}}
  \begin{itemize*}
    \item
    Developed an automated build system which allowed developers to see how
    their changes would affect a real build of the system. Reduced turnaround
    time, allowing anyone to easily run a build.
    \item
    Addressed user reported issues in Ruby/Rails and LUA, including hardening
    validations, and updating database migrations for old versions of the
    software.
    \item
    Designed a class/model structure for users and accesspoints, which allowed
    the back-end to represent clients of individual accesspoints.
  \end{itemize*}

\end{itemize}


\vspace*{\fill}


% Interests section.
\hrule
\begin{indentsection}{\parindent}
\begin{description*}
\item[Some Interests:]
  Teaching, Microelectronics, Woodworking, Music, Gaming, Skiing, Cats.
\end{description*}
\end{indentsection}


% I guess this is two pages now.
\clearpage


% Education section.
\hrule
\vspace{-0.4em}
\subsection*{Education}
\begin{itemize}
  \parskip=0.1em

  \item
  \headerrow
    {\textbf{Northeastern University}}
    {\textbf{Boston, MA}}
  \headerrow
    {\emph{College of Computer and Information Science}}
    {\emph{2011 -- 2016}}
    {\emph{Bachelor of Science in Computer Science}}
  \begin{itemize*}
    \item Relevant Courses Taken:
    \begin{itemize*}
      \item Programming Languages
      \item Special Topics in Programming Languages
      \item Compilers
      \item Systems and Networks,
      \item Computer Organization
      \item Software Development (aka HELL),
      \item Theory of Computation
      \item Algorithms and Data Structures
      \item Fundamentals of Computer Science 1 \& 2,
      \item Object Oriented Design,
      \item Artificial Intelligence
      \item Logic and Computation,
      \item Combinatorics
    \end{itemize*}
    \item Clubs \& Extracurriculars:
    \begin{itemize*}
      \item NU Hacks
      \item Hack Beanpot
      \item ACM
    \end{itemize*}
  \end{itemize*}
\end{itemize}


\vspace*{\fill}


% Metalink
\begin{center}
\definecolor{gray}{rgb}{0.5,0.5,0.5}
\footnotesize \LaTeX \ source available at
\url{https://github.com/nixpulvis/resume}
\end{center}


\end{document}
