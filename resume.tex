\documentclass[10pt,letterpaper]{article}
\usepackage[letterpaper,margin=3em]{geometry}
\usepackage{mdwlist}
\usepackage{color}
\pagestyle{empty}
\setlength{\tabcolsep}{0em}

% indentsection style, used for sections that aren't already in lists
% that need indentation to the level of all text in the document
\newenvironment{indentsection}[1]%
{\begin{list}{}%
  {\setlength{\leftmargin}{#1}}%
  \item[]%
}
{\end{list}}

% opposite of above; bump a section back toward the left margin
\newenvironment{unindentsection}[1]%
{\begin{list}{}%
  {\setlength{\leftmargin}{-0.5#1}}%
  \item[]%
}
{\end{list}}

% format two pieces of text, one left aligned and one right aligned
\newcommand{\headerrow}[2]
{\begin{tabular*}{\linewidth}{l@{\extracolsep{\fill}}r}
  #1 &
  #2 \\
\end{tabular*}}


\begin{document}

% Resume header.
\begin{center}
{\huge \textbf{Nathan Lilienthal}}\\\vspace{.2ex}
{\large github.com/nixpulvis $\circ$ nathan@nixpulvis.com $\circ$ 202-701-4368
$\circ$ Available January - June 2014}\vspace{-1ex}
\end{center}

% Education section.
\hrule
\vspace{-0.4em}
\subsection*{Education}

\begin{itemize}
  \parskip=0.1em

  \item 
  \headerrow
    {\textbf{Northeastern University}}
    {\textbf{Boston, MA}}
  \\
  \headerrow
    {\emph{College of Computer and Information Science}}
    {\emph{2011 -- Present}}
    {\emph{Bachelor of Science in Computer Science expected in 2016}}
  \begin{itemize*}
  \item Relevant Courses: Systems and Networks, Artificial Intelligence, Theory of Computation, Object Oriented Design,
  Computer Organization, Fundamentals of Computer Science 1 \& 2, Logic and Computation
   \item GPA: 3.125/4.000
  \end{itemize*}

\end{itemize}


% Misc section.
\hrule
%\vspace{-0.4em}

\begin{indentsection}{\parindent}
\hyphenpenalty=1000
\begin{description*}
  \item[Languages:]
  C, Java, JavaScript, Ruby, Python, Shell, LUA, Arduino, Racket
  \item[Systems:]
  Git, GitHub, Postgres, Redis, Rails, UNIX, Puppet
  \item[Interests:]
  Microelectronics, Espresso, Skiing, Woodworking (aspiring), Computer Architecture, Teaching
\end{description*}
\end{indentsection}


% Experience section
\hrule
\vspace{-0.4em}
\subsection*{Experience}

\begin{itemize}
  \parskip=0.1em

  \item
  \headerrow
    {\textbf{Bluesocket - Adtran}}
    {\textbf{Burlington, MA}}
  \\
  \headerrow
    {\emph{Software Developer}}
    {\emph{Jan 2013 -- Jun 2010}}
  \begin{itemize*}
    \item Developed an automated build system to allow developers to see how their
    changes would effect a real build of the system. Made turnaround time on
    changes faster, and allowed everyone to run a build.
    \item Fixed user reported issues in Ruby/Rails and LUA. Including hardening
    validations, and updating database migrations for old versions of the software.
    \item Designed a class/model structure for users and accesspoints to allow
    the backend to represent clients of individual accesspoints.
  \end{itemize*}

  \item
  \headerrow
    {\textbf{Northeastern University CCIS}}
    {\textbf{Boston, MA}}
  \\
  \headerrow
    {\emph{Tutor for Fundamentals of Computer Science 1}}
    {\emph{Fall 2012, Fall 2013}}
  \begin{itemize*}
    \item Assist in labs for the class by monitoring the classes progress,
    helping students with questions, and teaching concepts in new ways.
    \item Teach students during office hours about concepts of the course. Provide a
    new outlook on problems and guide students to their own answers.
    \item Grade homework, and provide feedback.
  \end{itemize*}

  \item
  \headerrow
    {\textbf{Homer Energy}}
    {\textbf{Boulder, CO}}
  \\
  \headerrow
    {\emph{Software Developer, Summer Intern}}
    {\emph{Summer 2013}}
  \begin{itemize*}
    \item Debugged a flex/flash application, cleaning up the UI and fixing
    issues.
    \item Developed an internal tool to view the Google Protocol Buffer
    used to pass values between all parts of the application. This allowed
    developers to quickly see what the values of program inputs and outputs were.
    \item Began a web based front-end for HOMER in Rails.
    This was the starting point for the starter edition of HOMER, and provided
    a proof of concept for how to integrate the HOMER API with a webserver.
  \end{itemize*}

\end{itemize}


% Projects section.
\hrule
\vspace{-0.4em}
\subsection*{Current Projects}

\begin{itemize}
  \parskip=0.1em

  \item 
  \headerrow
    {\textbf{Quadcopter}}
    {\textbf{3 Involved People}}
  \\
  \headerrow
    {\emph{Autonomus / Remote Controled Arial Drone}}
    {\emph{2013 -- Current}}

  \begin{itemize*}
    \item RC electronics, including 3-phase brushless motors, electronic speed
    controllers, and LiPo batteries.
    \item Dug deeper into Arduino and microelectronics circuits, connecting the
    motors to the self-made control board. The control board has a slew of sensors
    on it including the all important IMU.
    \item Learned a bit of 3D rotational math needed to understand and compute with
    data from the accelerometer and gyroscope to stabilize the craft.
  \end{itemize*}

\item 
  \headerrow
    {\textbf{NUACM Website}}
    {\textbf{5+ Involved People}}
  \\
  \headerrow
    {\emph{Marketing / Information Collection \& Display Website}}
    {\emph{2013 -- Current}}

  \begin{itemize*}
    \item Solely wrote the full backend in Rails for the website, with models for members
    of NUACM, payment dues, events held by the club, and more.
    \item Contributed to a large degree on the frontend, to display and collect information from members,
    and to promote the club.
    \item It's all on Github, except the secret server configs. github.com/nuacm/website
  \end{itemize*}

\end{itemize}


% Metalink
\begin{center}
\definecolor{gray}{rgb}{0.5,0.5,0.5}
\textcolor{gray}{\textit{Code for this resume is available on GitHub. github.com/nixpulvis/resume}}
\end{center}

\end{document}
