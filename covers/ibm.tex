\documentclass[10pt,letterpaper]{article}
\usepackage[letterpaper,margin=3em]{geometry}
\usepackage{mdwlist}
\usepackage{color}
\usepackage{varwidth}
\usepackage[hidelinks]{hyperref}

\DeclareUnicodeCharacter{03BB}{\ensuremath{\lambda}}

% Configuration.
\pagestyle{empty}
\setlength{\tabcolsep}{0em}

% indentsection style, used for sections that aren't already in lists
% that need indentation to the level of all text in the document
\newenvironment{indentsection}[1]
{\begin{list}{}
  {\setlength{\leftmargin}{#1}} \item[]
}
{\end{list}}

% opposite of above; bump a section back toward the left margin
\newenvironment{unindentsection}[1]
{\begin{list}{}
  {\setlength{\leftmargin}{-0.5#1}} \item[]
}
{\end{list}}
% format two pieces of text, one left aligned and one right aligned
\newcommand{\headerrow}[2]
{\begin{tabular*}{\linewidth}{l@{\extracolsep{\fill}}r}
  #1 &
  #2 \\
\end{tabular*}}


\addtolength{\oddsidemargin}{.75in}
\addtolength{\evensidemargin}{.75in}
\addtolength{\textwidth}{-1.5in}
\linespread{1.5}

\begin{document}

\begin{center}
    \huge \textbf{Nathan Lilienthal} \\
    \large
    \href{mailto:nathan@nixpulvis.com}{nathan@nixpulvis.com}
    $\circ$
    \url{https://nixpulvis.com}
    $\circ$
    \href{tel:12027014368}{+1-202-701-4368}
    \vspace{-0.2em}
\end{center}

\vspace{3em}

\noindent Dear reader,
\hfill
{\bf\today}
\vspace{1em}

I'm applying for the position of Technology Education Lead at IBM Research
because consuming, internalizing, and finally producing useful kernels of
knowledge from research is a passion of mine. I believe strongly that
information accessibility is one of the most crucial elements of good R\&D,
despite the necessary depth. I often hear fellow researchers brush off an
explanation due to overwhelming complexity, or fuzzy results. The great Richard
Feynman once said, ``I couldn't reduce it to the freshman level. That means we
really don't understand it.''

During my undergraduate years, I had the opportunity to TA for the freshman
Fundamentals of Computer Science course. In doing so I was able to test my
understanding through teaching quite rigorously. Every week I'd give a prepared
lecture to my lab section and assist students as they went through the
material. I also took it upon myself to write a JSON lab using a parser I had
written, so students might be inspired to take a look under the hood. You can
read the lab (please excuse the HTML to PDF conversion) and source code here if
you wish: \url{https://github.com/nixpulvis/parser-combinator}.

In my role as a research programmer at Northeastern Univeristy's Cybersecurity
and Privacy Institute, much of my work was to take knowledge from Programming
Language research and connect it to the work of my colleagues in Cryptography.
This included writing a large collection of examples, which would guide the
design of the future work. When I started I only had a surface level
understanding of Cryptography, so it was a crucial aspect of my job to be able
to read papers and extract meaningful and useful concepts quickly. As part of
this process, I kept notes which you can read some examples of starting here:
\url{https://nixpulvis.com/research/2019-10-13-achilles-8}.

In addition to my experience in academia, I have been part of a handful of
teams in industry. Software engineering requires technologies with clear and
concise examples and documentation so you can do your job reliably. On the flip
side, having a playground to experiment in is crucial to learning new tools. At
Apple, one project was to assist with large data sets for electrical engineers.
This is where I first learned how valuable a Jupyter notebook could be. Instead
of forcing the EEs to learn how to integrate with our full software stack, they
could experiment in Jupyter quickly and easily, without the risk of breaking
things. Meanwhile, we managed the database.

If I am accepted to this job at IBM, I'm confident my experience in both
academia and industry will be put to good use. My previous experience with both
software engineering practices and the pedagogical process makes me a strong
candidate. I am particularly eager to reach that freshman level of
understanding in Quantum Computing and divulge this knowledge to our potential
customers in an accessible way.

\vspace{1.5em}
\noindent Thank you for your consideration,\\
\noindent Nathan Lilienthal

\end{document}
